  %-----------------------------------------------
% DOCUMENT PACKAGES
%-----------------------------------------------
\documentclass[10pt]{article}
\usepackage[utf8]{inputenc}
\usepackage[T1]{fontenc}
\usepackage{graphicx}
\usepackage[margin=1.3in]{geometry}
\usepackage{hyperref}
\usepackage[french]{babel}
\usepackage[small, sc, bf, center]{titlesec}
\usepackage{listings}
\usepackage{amsmath, amssymb, mathtools}
\usepackage{cleveref}
\usepackage[table]{xcolor}
\usepackage{fancyhdr}
\usepackage{tikz}
\usepackage{tkz-graph}
\usepackage{csvsimple}
\usepackage{listings}
%\usepackage{subcaption}
%\usepackage{multicol}
%-----------------------------------------------
% DOCUMENT CONFIG
%-----------------------------------------------

% graphicx
\graphicspath{ {images/} }
% Add point after title number
\titleformat{\section}[block]{\sc\bfseries\center\large}{\thesection.}{0.5em}{}
\titleformat{\subsection}[block]{\sc\bfseries\center}{\thesubsection.}{0.5em}{}
\titleformat{\subsubsection}[block]{\sc\bfseries\center}{\thesubsubsection.}{0.5em}{}
% Page number reformat
\pagestyle{fancy}
\fancyfoot[C]{--~\thepage~--}
% Deactivate fancyhdr header
\renewcommand{\headrulewidth}{0pt}
\fancyhead{}
% tikz
\tikzstyle{vertex}=[circle, draw, inner sep=0pt, minimum size=6pt]
\newcommand{\vertex}{\node[vertex]}
\usetikzlibrary{arrows,petri,topaths,calc}
% listing style
\lstset{
frame=single,
basicstyle=\ttfamily\small,
numbers=left,
%numbersep=5pt,
%font=\ttfamily
}

%-----------------------------------------------
% DOCUMENT BODY
%-----------------------------------------------
\begin{document}
	
\begin{center}
	\textbf{Projet de FOSYMA\\[.5cm]Projet - Wumpus Multi-agent}\\[.5cm]
	\textit{B.Thanh Luong, Gualtiero Mottola}\\
	
	\includegraphics{logo}

\end{center}

\tableofcontents

\section{Introduction}
	Ce projet Consiste a développer une version multi-agent d'un jeu Fortement inspiré de "Hunt the Wumpus"  cette variante du jeu est définie de la façon suivante : un ensemble d'agents en coopération sont placé dans un environnement inconnu on pour mission d'explorer cet environnement et de récupérer un maximum de trésors qui sont disséminé dans cet environnement. Un agent Wumpus se trouve également dans l'environnement, il se déplace aléatoirement et a pour but de gêner l'exploration et la récupération des trésors.
	
\section{Présentation des Agents}
	Les Trois type d'agent utilisables pour récolter un maximum de Trésors sur la carte sont les suivants : les agents explorateurs qui n'ont pas la possibilité de récupérer des ressources, leur seul but est d'explorer la carte, des Agents Collecteurs qui on un sac a dos correspondant a un type de trésor (\texttt{TREASURE} ou \texttt{DIAMONDS}) et qui on une méthode permettant de récupérer ce type de trésor et le placé dans leur sac si celui ci n'est pas plein, On note que lorsque cette action est exécuté une partie du trésor est perdue. Enfin le dernier type d'agent, l'agent tanker qui ne peux pas ramasser de trésor mais a un sac a dos de capacité illimité, tous les agents collecteurs on la possibilité de donner leur trésors a l'agent tanker. Ce sons les quantité présentes dans L'agent tanker qui serons comptabilisé a la fin de l'exécution.

	\subsection{Comportement des Agents}
	les comportements de nos trois types d'agents sont tous implémentés sous la forme de \texttt{FSMBehaviours} qui sont des automates finis, Chaque état du FSM est un behaviour qui est exécuté selon l'ordre définit par l'utilisateur.nous allons décrire dans cette section le comportement principal de chaque agent, puis dans la section suivante la suite de behaviours qui leur permet de communiquer et qui est identique pour tout les types d'agents.  
	

\begin{lstlisting}
FSMBehaviour fsmBehaviour = new FSMBehaviour();
fsmBehaviour.registerFirstState(new ExploreBehavior(this),"Exp");
fsmBehaviour.registerState(new CheckMailBehavior(this),"Ckm");
fsmBehaviour.registerState(new RequestConnectionBehaviour(this),"Com");
fsmBehaviour.registerState(new SendMapBehaviour(this),"Smp");
fsmBehaviour.registerState(new ReceiveMapBehaviour(this),"Rmp");

fsmBehaviour.registerTransition("Exp","Ckm",1); //explore to check mail

fsmBehaviour.registerTransition("Ckm","Com",1); //check mail to start com
fsmBehaviour.registerTransition("Ckm","Smp",2); //check mail to send map

fsmBehaviour.registerTransition("Com","Rmp",1); //com to receive

fsmBehaviour.registerTransition("Smp","Rmp",1); // send to receive
fsmBehaviour.registerTransition("Smp","Exp",2); // send to exp

fsmBehaviour.registerTransition("Rmp","Exp",1); // receive to explore
fsmBehaviour.registerTransition("Rmp","Smp",2); // receive to send

addBehaviour(fsmBehaviour);
\end{lstlisting}
La classe offre des méthodes pour enregistrer les états et les transitions qui définissent l'ordre des behaviours
	
	
	
	\paragraph{Agent Explorateur}

	\paragraph{Agent Collecteur}
	
	\paragraph{Agent Tanker}{

	\section{Processus de Communication}

	\subsection{Les Processus}	
	send map et tout 	
	
	\subsection{Les Outils}
	ici dikstra par ex 
		
	\subsection{turcs pas implementé}

\section{Conclusion}



	
\end{document}